\documentclass[12pt]{article}

\usepackage{sbc-template}     % Modelo de relatório SBC
\usepackage[utf8]{inputenc}   % Codificação de entrada UTF-8
\usepackage[T1]{fontenc}      % Codificação de fonte: T1
\usepackage[brazil]{babel}    % Idioma português brasileiro
\usepackage{graphicx}         % Inclusão de imagens externas
\usepackage{minted}           % Destaque de sintaxe de código-fonte
\usepackage{caption}          % Legendas para imagens, tabelas, listagens, ...
\usepackage{hyperref}         % Referências em forma de links na saída em PDF

\title{Manipulador de Gramáticas Regulares e Livres de Contexto}

\author{Eduardo Weiland\inst{1}}
\address{Curso de Ciência da Computação -- Universidade de Santa Cruz do Sul (UNISC)\\
  Santa Cruz do Sul -- RS -- Brasil
  \email{eduardoweiland@mx2.unisc.br}
}

% ===========================================
% Configurações para listagens/código-fonte
% -------------------------------------------
\newcommand{\codefontsize}{\footnotesize}
\renewcommand\listingscaption{Listagem}
\providecommand*{\listingautorefname}{Listagem}

\newmintedfile{js}{linenos,numbersep=5pt,frame=leftline,framesep=2mm,fontsize=\codefontsize}

% http://tex.stackexchange.com/questions/12428/code-spanning-over-two-pages-with-minted-inside-listing-with-caption
\newenvironment{code}{\captionsetup{type=listing}}{}

% ===========================================
% Configurações de links na saída em PDF
% -------------------------------------------
\hypersetup{
  pdftitle={Manipulador de Gramáticas Regulares e Livres de Contexto},
  pdfauthor={Eduardo Weiland},
  pdfcreator={\LaTeX},
  pdfsubject={Manipulador de Gramáticas Regulares e Livres de Contexto (Trabalho de Linguagens Formais UNISC 2015/1)},
  pdfproducer={LaTeX},
  colorlinks=true,
  linkcolor=blue,
  citecolor=blue,
  filecolor=blue,
  urlcolor=blue,
  bookmarksdepth=4
}

\begin{document}

\maketitle

\begin{resumo}
  Este artigo relata as etapas do desenvolvimento de uma ferramenta para manipulação de gramáticas regulares e livres
  de contexto, desenvolvido na disciplina de Linguagens Formais da Universidade de Santa Cruz do Sul, no primeiro
  semestre de 2015.
\end{resumo}

%%%%%%%%%%%%%%%%%%%%%%%%%%%%%%%%%%%%%%%%%%%%%%%%%%%%%%%%%%%%%%%%%%%%%%%%%%%%%%%%%%%%%%%%%%%%%%%%%%%%%%%%%%%%%%%%%%%%%%%%

\section{Manipulador de Gramáticas}

Este artigo relata o processo de desenvolvimento de uma ferramenta para manipulação de gramáticas regulares e livres de
contexto. Essa ferramenta é um trabalho desenvolvido na disciplina de Linguagens Formais na Universidade de Santa Cruz
do Sul.

O ferramenta é dividida em duas partes: a primeira delas consiste em uma interface que permite criar interativamente uma
gramática do tipo regular ou do tipo livre de contexto; a segunda parte é um reconhecedor do tipo autômato finito,
capaz de reconhecer as sentenças geradas com a gramática criada na primeira parte.

O manipulador de gramática deve permitir a entrada de uma gramática qualquer pelo usuário, através de campos para a
definição dos símbolos não terminais e terminais, o símbolo de início de produção e todo o conjunto de produções. A
partir dessas entradas, é esperado que o programa verifique se a gramática entrada é válida, exiba o seu formalismo,
classifique-a na hierarquia de Chomsky e gere automaticamente duas sentenças para essa gramática.

O autômato finito que atua como reconhecedor deve ser representado através de uma tabela de transição de estados, que
deve ser editável. O autômato deve demonstrar passo-a-passo o processo de reconhecimento para as sentenças geradas pelo
manipulador de gramática.

%%%%%%%%%%%%%%%%%%%%%%%%%%%%%%%%%%%%%%%%%%%%%%%%%%%%%%%%%%%%%%%%%%%%%%%%%%%%%%%%%%%%%%%%%%%%%%%%%%%%%%%%%%%%%%%%%%%%%%%%

\section{Desenvolvimento}

A ferramenta foi desenvolvida como uma aplicação \textit{web} executada totalmente no lado cliente (\textit{clientside}).
Para isso, foi utilizada a linguagem de programação JavaScript em conjunto com algumas bibliotecas e outras ferramentas
de uso livre para facilitar a organização e a programação do trabalho.

%%%%%%%%%%%%%%%%%%%%%%%%%%%%%%%%%%%%%%%%%%%%%%%%%%%%%%%%%%%%%%%%%%%%%%%%%%%%%%%%%%%%%%%%%%%%%%%%%%%%%%%%%%%%%%%%%%%%%%%%

\section{Conclusão}

%%%%%%%%%%%%%%%%%%%%%%%%%%%%%%%%%%%%%%%%%%%%%%%%%%%%%%%%%%%%%%%%%%%%%%%%%%%%%%%%%%%%%%%%%%%%%%%%%%%%%%%%%%%%%%%%%%%%%%%%

\end{document}
