\documentclass[12pt]{article}

\usepackage{sbc-template}     % Modelo de relatório SBC
\usepackage[utf8]{inputenc}   % Codificação de entrada UTF-8
\usepackage[T1]{fontenc}      % Codificação de fonte: T1
\usepackage[brazil]{babel}    % Idioma português brasileiro
\usepackage{graphicx}         % Inclusão de imagens externas
\usepackage{minted}           % Destaque de sintaxe de código-fonte
\usepackage{caption}          % Legendas para imagens, tabelas, listagens, ...
\usepackage{hyperref}         % Referências em forma de links na saída em PDF

\title{Manipulador de Gramáticas Regulares e Livres de Contexto}

\author{Eduardo Weiland\inst{1}}
\address{Curso de Ciência da Computação -- Universidade de Santa Cruz do Sul (UNISC)\\
  Santa Cruz do Sul -- RS -- Brasil
  \email{eduardoweiland@mx2.unisc.br}
}

% ===========================================
% Configurações para listagens/código-fonte
% -------------------------------------------
\newcommand{\codefontsize}{\footnotesize}
\renewcommand\listingscaption{Listagem}
\providecommand*{\listingautorefname}{Listagem}

\newmintedfile{js}{linenos,numbersep=5pt,frame=leftline,framesep=2mm,fontsize=\codefontsize}

% http://tex.stackexchange.com/questions/12428/code-spanning-over-two-pages-with-minted-inside-listing-with-caption
\newenvironment{code}{\captionsetup{type=listing}}{}

% ===========================================
% Configurações de links na saída em PDF
% -------------------------------------------
\hypersetup{
  pdftitle={Manipulador de Gramáticas Regulares e Livres de Contexto},
  pdfauthor={Eduardo Weiland},
  pdfcreator={\LaTeX},
  pdfsubject={Manipulador de Gramáticas Regulares e Livres de Contexto (Trabalho de Linguagens Formais UNISC 2015/1)},
  pdfproducer={LaTeX},
  colorlinks=true,
  linkcolor=blue,
  citecolor=blue,
  filecolor=blue,
  urlcolor=blue,
  bookmarksdepth=4
}

\begin{document}

\maketitle

\begin{resumo}
  Este artigo relata as etapas do desenvolvimento de uma ferramenta para manipulação de gramáticas regulares e livres
  de contexto, desenvolvido na disciplina de Linguagens Formais da Universidade de Santa Cruz do Sul, no primeiro
  semestre de 2015.
\end{resumo}

%%%%%%%%%%%%%%%%%%%%%%%%%%%%%%%%%%%%%%%%%%%%%%%%%%%%%%%%%%%%%%%%%%%%%%%%%%%%%%%%%%%%%%%%%%%%%%%%%%%%%%%%%%%%%%%%%%%%%%%%

\section{Manipulador de Gramáticas}

%%%%%%%%%%%%%%%%%%%%%%%%%%%%%%%%%%%%%%%%%%%%%%%%%%%%%%%%%%%%%%%%%%%%%%%%%%%%%%%%%%%%%%%%%%%%%%%%%%%%%%%%%%%%%%%%%%%%%%%%

\section{Desenvolvimento}

%%%%%%%%%%%%%%%%%%%%%%%%%%%%%%%%%%%%%%%%%%%%%%%%%%%%%%%%%%%%%%%%%%%%%%%%%%%%%%%%%%%%%%%%%%%%%%%%%%%%%%%%%%%%%%%%%%%%%%%%

\section{Conclusão}

%%%%%%%%%%%%%%%%%%%%%%%%%%%%%%%%%%%%%%%%%%%%%%%%%%%%%%%%%%%%%%%%%%%%%%%%%%%%%%%%%%%%%%%%%%%%%%%%%%%%%%%%%%%%%%%%%%%%%%%%

\end{document}
